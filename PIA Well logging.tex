\documentclass[10pt,a4paper]{report}
\usepackage[utf8]{inputenc}
\usepackage{amsmath}
\usepackage{amsfonts}
\usepackage{amssymb}
\usepackage{makeidx}
\usepackage{graphicx}
\usepackage{lmodern}
\usepackage{kpfonts}
\usepackage{fancyhdr}
\usepackage[left=2cm,right=2cm,top=2cm,bottom=2cm]{geometry}
\rhead[z1]{Registros Geofísicos de Pozos}
\rfoot[e1]{Jair Medina Ortiz}
\fancypagestyle{plain}{
\fancyhead[L]{K1}
\fancyhead[C]{K2}
\fancyhead[R]{K3}
\fancyfoot[L]{L1}
\fancyfoot[C]{L2}
\fancyfoot[R]{L3}
\renewcommand{\headrulewidth}{0.5pt}
\renewcommand{\footrulewidth}{0.5pt}
}

\pagestyle{fancy}





\begin{document}

%Introducción

\section*{Registros Geofísicos de Pozos}
Para poder determinar los límites de un yacimiento, capacidad de almacenamiento, valores de permeabilidad, contenido de hidrocarburos y valor económico. Para evaluar las posibilidades de producción:
\\ 1.- Núcleos
\\ 2.- Mud logging
\\ 3. Registros de wireline
\\\\ Que son los registros geofísicos ?
\\ Es una medida versus profundiad de cualquier característica de la formación rocosa (Serra, 1988).
\\ Un simple record registrado en una larga hoja donde una descripción física de las rocas es medida en el pozo que ha sido perforado. Registros de wireline son mediciones realizadas a tráves de una herramienta la cual es bajada mediante un cable y la información es transmitida a tráves de este a una estación móvil en superficie. Existen dos tipos de registros wireline:
\\ 1. Registros de Agujero descubierto 
\\ 2. Registros de Agujero entubado
\\\\ La mayoria de los registros son en agujero descubierto donde pueden ser tomadas las mediciones directamente del pozo. Los registros en agujero entubado son menos comúnes y tienen que ser ajustados por ciertas irregularidades en la geometría y rugosidad de las paredes del pozo.
\\\\ Mediciones:
\\ $\bullet$ Porosidad
\\ $\bullet$ \; Saturación
\\ $\bullet$ \; Tipo de hidrocarburo
\\ $\bullet$ \; Ambiente sedimentario
\\ $\bullet$ \; Litología
\\ $\bullet$ \; Tiempo de tránsito
\\\\ Los registros geofísicos :
\\\\ Natural o Espontáneo
\\ 1. Potencial Espontáneo
\\ 2. Rayos Gamma
\\ 3. Temperatura
\\\\ Propiedades físicas medidas induciendo una respuesta a la formación:
\\ a. Resistividad
\\ b. Conductividad
\\ c. Indice de hidrógeno
\\ d. Captura de neutrones
\\ e. Decaimiento termal de neutrones
\\ f. Absorción foto eléctrica
\\ g. Tiempo de relajamiento del spin del protón
\\ h. Velocidad Acustica
\\ i. Propiedades Mecánicas


%Rayos Gamma

\newpage
Registro de Rayos Gamma
\\ El registro de Rayos gamma usa un contador de centelleo impulsando ondas electromágneticas de alta energía que son emitidos espontáneamente por algunos elementos radioactivos que permite hacer una medición de la radioactividad natural del potasio, torio y uranio en lasrocas a lo largo del pozo. Muchas formaciones arcillosas como la lutita y limolita presentan radioactividad.
\\ Muchas lutitas o limolitas pueden reflejar en la respuesta de este registro altos valores de radioactividad. Areniscas y calizas son rocas con un potencial alto para almacenar hidrocarburos y reflejan valores promedio o bajos en este registro. Un registro gamma ray es relativamente no costoso y puede ser empleado en agujero entubedo y descubierto. 
\\ Un registro espectral es una variación al gamma ray normal que permite identificar la fuente de radiación (potasio, torio y uranio) Este ayuda para distinguir la fuente de la radiación y diferenciar entre lutitas y areniscas. Ambos registros se usan para localizar zonas potenciales de almacenamiento de hidrocarburos. La arcillosidad, un estimado del contenido de arcilla en una formación puede ser estimada con el registro gamma ray.
\\\\El nivel de gamma ray se registra en unidades API en escala de 0-150 API.
\\  $$\includegraphics[scale=2.2]{gammaray1}$$ 
\\ Fig. 1 Valores Tipicos de los registros gamma ray, extracción del golfo de México. 100 formación limpia y 100 formación arcillosa. 
\\\\ La radioactividad natural refleja los elementos radioactivos que se concentran en los clastos de las rocas. Los radioisótopos que se usan como referencia son:
\\ 1. Potasio (K$^40$) vida media de $1.3x10^9$ años
\\ 2. Uranio (U$^236$) vida media de $1.4x10^10$ años
\\ 3. Torio (Th$^232$) vida media de $4.4x10^9$ años
\\ Ya que estos radioisótopos tienen una larga vida y su decaimiento produce una gran cantidad de rayos gamma.
\newpage
Aplicaciones: 
\\ a) Cálculo y estimación de la porosidad.
\\ b) Identificación de minerales radíoactivos.
\\ c) Indicador litológico .
\\ d) Correlación entre pozos.
\\ e) Evaluación de contenido arcilloso.
\\ ) La relación Th/U detecta los niveles de ceniza.
\\\\ Efectos en el registro
\\ 1. \textbf{Tamaño de Pozo} 
\\ a) 1 A mayor diámetro del pozo menor la intensidad de los rayos gamma en llegar a la sonda.
\\ 2. \textbf{Efectos del lodo}
\\ a) Absorbe un pequeño porcentaje de la radiación y reduce la amplitud del registro, A menos que el diámetro del pozo sea muy grande (mas de 24'') este efecto es despreciable.
\\ b) Arcilla contenida en el lodo puede aumentar en la lectura del registro. Si el lodo es uniforme este afecta uniformemente en la lectura y si el contenido arcilloso en el lodo se ha colocado en el fondo habrá un incremento en la toma de datos en la porción del registro, por lo que incluso las arenas limpias muestran ligera radioactividad en el registro, efectos que se debe tener en cuenta en la interpretación del registro.
\\ 3. \textbf {Efectos de la tubería de revestimiento}
\\ a) La mayoría de los rayos gamma emitidos a la formación pueden penetrar la tubería de revestimiento sin embargo la intensidad puede ser reducida a un cuarto.





%Neutron
\newpage
Esta medida se basa en el contenido de hidrógeno de la formación. En yacimientos el hidrógeno puede representar presencia de agua o petróleo. Los neutrones son partículas eléctricamente neutrales de la misma masa que el átomo de hidrógeno. Las fuentes usadas en los registros neutrón son combinaciones de minerales como Americio (Am) y Berilio (Be). Los neutrones dejan la fuente con alta energía cuando chocan con núcleos de hidrógeno, lo cual significa que los detecores diseñados para detectar neutrones con baja energía lo hacen despues que estos colisionan con átomos de minerales de la fromación, Y en unidades de tasa de conteo. Tasa de conteo alta demostrará pocos átomos de hidrógeno, lo cual significa que la formación es de baja porosidad. Al contrario una tasa de conteo bajo reflejará alta cantidad de átomos de hidrógeno, lo cual significa que la formación es de alta porosidad. Sensitividad de la herramienta proviene de la candidad de atomos de hidrógeno presentes en la zona de estudio.Uso principal es la determinación de la porosidad de la formación.
\\\\ Hay 3 tipos principales de herramienta neutrón:
\\ 1. Rayos gamma/ Herramienta neutrón (GNT)
\\ 2. Herramienta porosidad neutrón lateral (SNP)
\\ 3. Registro neutrón compensado (CNL)
\\
\\\\ Efectos en el registro
\\ 1. \textbf{Efecto del Gas} 
\\ La presencia de hidrocarburios liquidos (petróleo) no afecta la respuesta de la herramienta como tal y tiene aproximadamente el mismo índice de hidrógeno como el agua fresca. Sin embargo el hidrocarburo gas, tiene mucho menor índice de hidrógeno por su baja densidad y su presencia bajo estima las respuestas en lecturas de porosidad.
\\ 2. \textbf{Efectos de la Lutita}
\\ La Lutita contiene arcilla que tiene un valor significante de moléculas de agua en la superficie. Esto incrementa el índice de hidrógeno en la formación aun cuando valores bajos de porosidad pueden dar erroneamente valores altos de porosidad en las lecturas debido a la presencia de la arcilla.
\\ 3. \textbf {Efectos del cloro}
\\ Cloro es un buen absorbente de neutrones y puede dar resultado de sobre estimaciones de la lectura en porosidad por la presencia en el fluido en la formación y fluido de perforación.
\\\\ Pérdida de nergía neutrón puede ser relacionado con la porosidad porque en formaciones porosas, hidrógeno estáconcentrado en los fluidos saturando los poros como tambien la presencia de agua de formación, hidrocarburos o fluido de perforación. Los yacimientos que tienen en sus poros el gas pueden ser bajo estimados debido a los efectos del gas presentes en estos registros.
\\\\Aplicaciones: 
\\ a) Cálculo y estimación de la porosidad.
\\ b) Identificación de zonas altamente porosas.
\\ c) Indicador litológico. 
\\ d) Correlación entre pozos.
\\ e) Relación con registro densidad.
\newpage
$$\includegraphics[scale=0.5]{neutron1}$$
\\ Fig. Las unidades pueden estar en porcentaje (a menudo escritas como "unidades de porosidad" o "p.u.") o en unidades decimales (cero a uno), como en este ejemplo.
\\\\ Los neutrones rápidos experimentan una pérdida progresiva de energía.Comúnmente el elemento que absorbe estos neutrones de movimiento lento es el cloro, contenido en la salinidad del agua intersticial.Al absorber un neutrón, el núcleo emite un rayo gamma con una energía característica. Las herramientas antiguas tenían un solo detector y medían la cantidad de rayos gamma de captura o d eneutrones epitermales. Los dispositivos modernos tienen dos detectores de neutrones ubicados a poca distancia de la fuente para compensar los efectos del pozo.Debido a que la reducción en el flujo de neutrones está controlada en tal medida por la concentración de hidrógeno en la formación, la medición puede usarse para la estimación de la porosidad en litologías de yacimientos típicos.










%Densidad
\newpage 
El registro de densidad mide la densidad de la formación y la relaciona con la porosidad. Una fuente radioactiva emite radiación gamma hacia la formación, la cual interacciona con el efecto compton.
\\\\Aplicaciones: 
\\ a) Determinación de la porosidad.
\\ b) Identificación de minerales en depositos evaporíticos.
\\ c) Detección de gas .
\\ d) Determinación de la densidad de los hidrocarburos.
\\ e) Correlación entre pozos.
\\\\ En las formaicones de baja densidad (alta porosidad) se registran mas conteos de rayos gamma. En la medida que la densidad se incrementa (porosidad decrece), menos conteos de rayos gamma pueden ser detectados.
\\ Las presiones anormales afectan los registros de densidad, lo normal es que la densidad aumente con la profundidad sin embargo en zonas sobrepresurizadas esta tendencia cambia:
\\ $\bullet$ Determinación de inconformidades.
\\ $\bullet$ Detección de sobrepresiones.
\\ $\bullet$ Reconocimiento de fracturas.
\\ $\bullet$ Contenido de materia orgánica.
\\\\ La herramienta de densidad consiste típicamente en una fuente radioactiva de rayos gamma (Cesio, 137) que está montada en una colchoneta de caucho que se presiona contra la formación por la herramienta.Es otra aplicación de los rayos gamma para obtener datos sobre las formaciones.El Enjarre y el frente de la herramienta tienen grandes efectos en la toma de lectura. La combinación de los registros neutrón y densidad pueden dar una buena fuente de información para la estimación y cálculo de porosidad, sobre todo en formaciones con una litología compleja. Esta herramienta de registro emite rayos gamma que colisionan con los electrones de la formación y se dispersan. Un detector fijado en una distancia considerable registra el numero de rayos que gamma que regresan y es un indicador proporcional de la densidad.
\\ $$\\  \includegraphics[scale=0.6]{densidad1}$$
\\ Fig. La curva de densidad aparente y su escala de densidad convencional entre 2 y 3 gramos por centímetro cúbico. Este rango es para la mayoría de las litologías sedimentarias.






%Sonico
\newpage
El registro sonico o de velocidad acustica (AVL)mide la velocidad de sonido a traves de cada capa de roca en el pozo.La herramienta de registro tiene un transmisor de sonido en el tope de la herramienta..Un impulso de sonido es emitido por el transmisor y es registrado en los dos recibidores colocados a lo largo de la herramienta.Un impulso de sonido es emitido y el tiempo que toma el sonido para viajar de un recibidor a otro a traves de las rocas es registrado y medido en unidades de microsegundos por pie de formacion. Esta velocidad es llamada el intervalo de tiempo de transito o $\Delta t$ de la roca.
\\ La siguiente tabla 1. muestra los rangos mas comunes de velocidades de sonido a traves de las rocas sedimentarias, agua o gas natural. Las lutitas tienen la velocidad sonica mas lenta, areniscas tienen mayores valores de velocidad y las calizas y dolomias tienen las mayores velocidades sonicas. En cada tipo de rocas sedimentarias porque la velocidad de sonido a traves del gas y liquidos es menor a la de los solidos como las rocas. Entre mas porosa sea una roca, O entre mas gas o liquido contenga menor sera su velocidad sonica.
$$\begin{tabular}{ c c c c}
Table 1 & Velocidad (ft/s) & m/s & t \\
Lutita & 7,000 - 17,000 & 2,134-5,182 & 144-59 \\
Arenisca & 11,500 - 16,000 & 3,505-4,877 & 87-62 \\
Caliza & 13,000 - 18,500 & 3,962-5,639 & 77-54 \\
Dolomia & 15,000 - 20,000 & 4,475-6,096 & 67-50 \\
Gas natural & 1,500 & 456 & 667 \\
Agua & 5000 & 1524 & 200 \\
\end{tabular}$$ 
\\
Los registros acústicos más usados son:
\\ a) Sónico de porosidad, Principio acústico es usado en un pozo sin tubería, es decir, sobre la litología. Al pasar por las calizas, arenas, etc. Cambia su velocidad de recepción.
\\ b) Sónico de cementación, Se utiliza para pozos ya entubados y el principio lo utiliza para verificar la fijación o los vacíos entre el cemento, la tubería de revestimiento y la formación.
\\ c) Sónico Digital,La forma en que trabaja, el tipo de transmisión de datos es diferente, las pérdidas por el cable y por frecuencia o ruidos, se eliminan, es decir no hay error en la información. 
\\ d) Sónico Bipolar,Como su nombre lo indica contiene dos polos, las características de los transmisores son diferentes. Este tipo contiene más receptores y por tanto pueden determinar otro tipo de parámetros por medio de interpretaciones que se llevan a cabo en un procesador en la superficie.
\\ $$\\  \includegraphics[scale=0.9]{sonico1}$$ 
\\ En la medición sónica más básica obtenida con cable, un transductor acústico emite una señal sónica, oscilante entre 10 y 30 kHz, que es detectada en dos receptores situados en el pozo, en dirección hacia la superficie. El tiempo entre la emisión y la recepción se mide para cada receptor, y se sustrae con el fin de obtener el tiempo de viaje en el intervalo comprendido entre los dos receptores. Si los receptores están separados por una distancia de dos pies, este tiempo se divide por dos para obtener el tiempo de tránsito de intervalo, o lentitud, de la formación. Este tipo de medición se conoce también como detección del primer movimiento. 





%Resistividad
\newpage
Un registro de la resistividad de la formación, expresado en ohm-m. La resistividad puede adoptar una amplia gama de valores y por consiguiente por razones de conveniencia se presenta generalmente en una escala logaritmica comprendida, por ejemplo entre 0.2 y 2,000 ohm-m, El registro de resistividad es fundamental en la evaluación de formaciones porque los hidrocarburos no conducen la electricidad, en tanto que las aguas de formación si lo hacen. Por consiguiente existe una gran diferencia entre la resistividad de las rocas saturadas con hidrocarburos y con agua de formación. Existe una gran diferencia entre la resistividad de las rocas.Minerales de arcilla y otros, como la pirita que son conductores de electricidad y reducen la diferencia. Son registros inducidos.La resistividad es la capacidad que tiene la roca de oponerse al paso de corriente electrica inducida y el inverso de la conductividad.
\\ La resistividad depende de la sal disuelta en los fluidos presentes en los poros de las rocas.Proporciona evidencia del contenido de fluidos en las rocas.Proporciona evidencia del contenido de fluidos en las rocas.Si los poros de una formación contienen agua salada presentará alta conductividad y por lo tanto la resistividad será baja,pero si estan llenos de petróleo o gas, presentará baja conductividad y por lo tanto de resistividad alta.Las rocas compactas poco porosas como las calizas masivas poseen resistividades altas.
\\ Existen dos tipos principales de perfiles de resistividad: El perfil lateral (laterolog) y el perfil de inducción (Induction Log).El perfil lateral se utiliza en (lodos conductivos y lodos salados) y el eprfil de inducció se utiliza en lodos resistivos (lodo fresco o base aceite).
\\\\Dentro de los perfiles de resistividad de inducción tenemos:
\\a) SFL Spherical Induction Log.Para profundidades someras (0.5-1.5).Mide la resistividad de la zona lavada (Rxo)
\\ b)MIL=LIM=Medium Induction Log, distancias medias (1.5-3)
\\ c) DIL=ILD=Deep Induction Log.Para profundidades mas de 3'' y mide la resistividad de la formación (Rt)
\\\\ Dentro de los registros de resistividad lateral
\\ a) MSFL = microspheric laterolog. Para las proximidades (1 y 6').Lee la resistividad de la zona lavada (Rxo)
\\ b) MLL = Microlaterolog, proximidad (1 a 6')
\\ c) SLL = Someric Laterolog.Profundidad Somera (0.5-1.5')
\\ d) DLL = Deep Laterolog.Para profundidades de mas de 3' y miden la resistividad de la formación.
 \\\\ Se lee de izquierda a derecha, en escala logarítmica.La unidad de medida para los perfiles de resistividad es el ohm-m.Los registros de resistividad se usan para estimar contactos agua-petróleo, para calcular la resistividad de agua de formación (Rw) y la resistividad verdadera de la formación(Rt).Izquierda a derecha.
 \\ $$\\  \includegraphics[scale=0.6]{resistividad1}$$ 
 





%Ecuaciones
\newpage
Ecuaciones de saturación de agua
\\ Archie
\\ $$\displaystyle Sw=\sqrt[n]{\dfrac{a \; Rw}{\phi^m \; Rt}}$$
\\ Simandoux
\\ $$\displaystyle Sw=\left(\dfrac{0.4 \; Rw}{\phi^2}\right)\left(\sqrt{\left(\dfrac{Vshale}{Rsh}\right)^2} + \left(\dfrac{5 \; \phi^2}{Rt \; Rw}\right)-\left(\dfrac{Vshale}{Rsh}\right)\right)$$
\\ Indonesia
\\ $$\displaystyle Sw=\left(\left(\dfrac{Vsh^{2-Vsh}}{Rsh}\right)^{1/2}+\; \left(\dfrac{\phi_e^m}{Rw}\right)^{1/2} \; Rt\right)^{-1/n}$$
\\ Rayos gamma
\\ $$\displaystyle Vcl=\dfrac{GRlog-GRmin}{GRmax-GRmin}$$ 
\\ Modelo de Clavier
\\ $$\displaystyle Vcl=1.7-\left[3.38-(GRlog+0.7)^2\right]^{1/2}$$
\\ Modelo de Steiber $$\displaystyle Vcl=0.5\left(\dfrac{GRlog}{1.5 + ISh}\right)$$
\\ ISh = Indice de arcillosidad registrado a partir de la lectura
\\ Para el cálculo de K se utilizó las ecuaciones de Timur,Tixier y Coates, las cuales se describen a continuación:
\\ Timur
\\ $$K=8.58102 \dfrac{\phi^{4.4}}{Swc^2}$$
\\ Tixier
\\ $$K^{1/2}= \dfrac{250 \; \phi_e^3}{Swi}$$
\\ Morris Biggs
\\ $$k = 52.5 \left(\dfrac{\phi^3}{Swc}\right)^2$$
\\ Ecuaciones Porosidad
\\ Porosidad Densidad
\\ $$DPHI = \dfrac{RHOBmat-RHOBlog}{RHOBmat-RHOBfl}$$
\\Porosidad Sonica
\\ $$SPHI = \dfrac{DTlog-DTmat}{DTfl-DTmat}$$
\\ Densidad Neutron Porosidad
\\ $$\phi_{DN}=\dfrac{\phi_D-\phi_N}{2}$$
\\ Porosidad efectiva
\\ $$PHIE = PHIT \; (1-Vcl)$$



%imagenes
  Los datos obtenidos a partir de los registros geofísicos (rayos gamma, resistividad, densidad, neutrón y sónico) se registran
 en una tabla de Excel, El codigo que se realizó en el programador de Matlab nos permite hacer un cálculo eficaz de la saturación de agua, porosidad, volumen de lutita y permeabilidad, con los resultados se puede hacer una interpretacion cuantitativa y cualitativamente  así observando las zonas mas deseables a estudiar por los resultados que pueden indicar presencia de hidrocarburos. 
\\ \includegraphics[scale=0.6]{datos1} 
 \includegraphics[scale=0.6]{datos3} 
\\ \includegraphics[scale=0.6]{datos2} 
 \includegraphics[scale=0.7]{datos4} 









\newpage
Codigo de Matlab a continuación
\\ \includegraphics[scale=1.5]{codigo1} 
\newpage
\includegraphics[scale=1.5]{codigo2}
\newpage
\includegraphics[scale=1.5]{codigo3}
\newpage
\includegraphics[scale=1.5]{codigo4}
\newpage
\includegraphics[scale=1.5]{codigo5}
\newpage
\includegraphics[scale=1.5]{codigo6}
\newpage
\includegraphics[scale=1.5]{codigo7}
\newpage
\includegraphics[scale=1.5]{codigo8}
\newpage
\includegraphics[scale=1.5]{codigo9}
\newpage
\includegraphics[scale=1.5]{codigo10}
\newpage
\includegraphics[scale=1.5]{codigo11}






 
\newpage
Los 5 tracks y su representación se pueden determinar de la siguiente manera
\\ $$\includegraphics[scale=0.45]{registro1}$$
\\El código MATLAB nos imprime la tabla tambien para su analisis cuantitativo permitiendo así un análisis detallado
\\ $$\includegraphics[scale=0.7]{titulo1}$$
\\ $$\includegraphics[scale=0.7]{tabla1}$$


\newpage
A continuación seleccionamos dos evaluaciones por via cuantitativa y via cualitativa para hacer un análisis mas detallado de lass posibles zonas a desarrollar en la búsqueda de hidrocarburos
\begin{center}
\includegraphics[scale=0.5]{evaluacion1}
\end{center}
Conclusiones por Zonas de interés
\\ $$\includegraphics[scale=0.6]{titulo1}$$
\\ $$\includegraphics[scale=0.6]{evaluacion2}$$
\\ La primer zona de interés designada por un análisis cuantitativo y es a la profundidad de 1432 a 1433 metros de profundidad y se observa que los  valores de porosidad total es propicia a tener buenas acumulaciones de hidrocarburo, los volúmenes de lutita son bajos y eso nos habla de una formación limpia dando como resultado buenos valores de porosidad efectiva,Sin embargo los valores de saturación de agua son altos y esto elimina la posiblidad donde podamos tener una posible acumulación de petróleo economicamente rentable para designarlo como zona productora.
\newpage
$$\includegraphics[scale=0.6]{titulo1}$$
\\ $$\includegraphics[scale=0.6]{evaluacion3}$$
\\ La segunda zona de interés cumple el interválo de 1440 a 1441 metros y tenemos valores de porosidad total ligeramente menores a la primer zona de análisis e igual un volumen de lutita ligeramente mayor en promedio sin embargo nuestros valores de saturación de agua drasticamente descienden a valores por debajo del 50 por ciento de saturación lo cual nos da una buena información de una presencia de hidrocarburos saturando el volumen de roca que por sus valores de porosidad permiten almacenar dando asi una cantidad y tal ez calidad de designarla como zona productora
\\ $$\includegraphics[scale=0.6]{titulo1}$$
\\ $$\includegraphics[scale=0.6]{evaluacion4}$$
\\ \\ La tercera zona de interés que comprende una profundidad 1445-1447 metros de  y es designada por un análisis cuantitativo y tenemos valores de porosidad total muy similares a las zonas de interes pasadas,Un volumen de lutita bajo lo cual tambien nos indica una formación libre de arcilla,Los valores de saturación de agua son por debajo del 50 por ciento de saturación lo cual nos da una buena información de una presencia de hidrocarburos saturando el volumen de roca que por sus valores de porosidad permiten almacenar dando asi una cantidad y tal ez calidad de designarla como zona productor. Junto con la segunda y la tercera se puede empezar a desarrollar estudios mas detallados todo en pro de reducir la incertidumbre de las profundidades a estudiar y poder así hacer un plan de evaluación mas sofisticada. Con estas poderosas herramientas como MATLAB podemos hacer un cálculo mas eficiente y ágil donde los ingenieros pueden realizar sus estudios con mas detalle y es la herramienta que nosotros como estudiantes utilizamos para el análisis de este pozo.
\end{document}